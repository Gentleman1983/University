%% LyX 2.0.0 created this file.  For more info, see http://www.lyx.org/.
%% Do not edit unless you really know what you are doing.
\documentclass[a4paper,ngerman]{scrartcl}
\usepackage[T1]{fontenc}
\usepackage[latin9]{inputenc}

\makeatletter

%%%%%%%%%%%%%%%%%%%%%%%%%%%%%% LyX specific LaTeX commands.
\pdfpageheight\paperheight
\pdfpagewidth\paperwidth


\makeatother

\usepackage{babel}
\begin{document}

\section{VaR aus historischer Simulation}

Es gilt:

$VaR_{5d}^{99\%}=\$(-160.087,78);\quad VaR_{10d}^{99\%}=\$(-226.398,30)$

Somit ist der Verlust in 5 Tagen zu 99\% nicht h�her als ca. $\$160.087,78$
und in 10 Tagen als ca. $\$226.398,30$.


\section{Modellerweiterung}

Die g�ngigste Alternative zur historischen Simulation ist der Model-Building-Ansatz,
auch bekannt als Varianz-Covarianz- oder Delta-Normal-Ansatz. Dieser
basiert auf dem VaR-Modell von J.P. Morgan und beschreibt die Risikofaktoren
durch eine Covarianzmatix unter der Annahme einer multivariaten Normalverteilung
der �nderungen der Risikofaktoren. Aufgrund der Beschr�nkung auf Abbildung
linearer Zusammenh�nge zwischen Risikofaktoren und Marktpreisen ist
es schwierig in diesem Modell nicht-lineare Zusammenh�nge wie Optionen
zu bewerten. Eine L�sung dieses Problems wird durch die Cornish-Fisher-Approximation
angeboten, auf die wir hier nicht gesondert eingehen.

Vorteile dieses Ansatzes liegen in der einfachen Berechenbarkeit dieses
Ansatzes. Hierbei wird ein erwarteter Return aller Faktoren von 0,
sowie - wie oben schon gesagt - f�r alle Faktoren eine multivariate
Normalverteilung angenommen, wobei die Returns eines jeden beliebigen
Portfolios normalverteilt sind. Anschlie�end werden aus den historischen
Daten die Returns aus den Wert�nderungen berechnet und die Varianz-Covarianz-Matrix
f�r alle Faktoren aufgestellt. Anschlie�end wird die Portfoliovarianz
bestimmt und aus dieser - dank der Annahme der Normalverteilung -
der VaR bestimmt.


\section{VaR aus Model-Building}

Es gilt:

$VaR_{5d}^{99\%}=\$(-158.079,08);\quad VaR_{10d}^{99\%}=\$(-223.557,58)$

Somit ist der Verlust in 5 Tagen zu 99\% nicht h�her als ca. $\$158.079,08$
und in 10 Tagen als ca. $\$223.557,58$.

Da wir keine Positionen in Optionen haben, bietet sich der Model-Building-Ansatz
zur Bewertung an.


\section{Options im Portfolio}

Hier stellt sich das Problem, der fehlenden Linearit�t der Optionen
zu ihrem Underlying. Dies kann zwar �ber lineare Approximation, die
Nutzung von delta und die Abbildung auf ein Portfolio aus delta-Underlying
und risikofreiem Instrument ein wenig abgemildert werden, aber alles
in allem sollte das Portfolio ebenfalls den selben Preis wie die Option
habn.


\section{DUR{*}, Conv, relative price change}

$DUR^{*}=(-\frac{\frac{\Delta P}{P}}{\Delta r})=(-\frac{\frac{1.160,1253765-1.159,0974311}{1.159,6742320}}{0,0002})=(-4,4320438)$

$DV01(r)=DUR^{*}*P*0,0001=(-0,5139727)$

$DV01(r+1bp)=DUR**P(r+1bp)*0,001=(-0,5141726)$

$Conv=\frac{1.000.000}{P}*(DV01(r)-DV01(r-1bp))=0,1723760$

$P_{\Delta}=DV01(r)*(-200)=\$102,79454$

Bei einem Zinsanstieg um $200$ Basispunkte, steigt der Wert unserer
Anlage um $\$102,79454$.


\section{Duration of the fund}

$DUR_{E}=\frac{\$500.000}{\$4.000.000}*0+\frac{\$3.500.000}{\$4.000.000}*6-\frac{\$1.500.000}{\$3.000.000}*5-\frac{\$1.500.000}{\$3.000.000}*10=(-2,25)$yrs.


\section{Immunization by Interest Rate Swap}

Unsere aktuelle Equity-Laufzeit betr�gt $DUR_{E}=-2,25$. Da $DUR_{E}<0$
gilt, m�ssen wir um die Anlagen zu immunisieren, die Laufzeit erh�hen.
Hierzu steht uns eine Anlage mit $5$ Jahren und eine mit $0,5$ Jahren
Laufzeit zur Verf�gung. Um die Laufzeit zu erh�hen, w�hlen wir die
Rolle des Floating-Rate-Payers, also wir m�chten gerne die Fixed Rate
erhalten.

Es gilt:

$DUR_{Swap}=(-5)+0,5=(-4,5)$

$\Delta DUR_{E}=\Delta DUR_{Swap}*\frac{x}{P_{E}}$

Umgestellt ergibt sich:

$x=\frac{\Delta DUR_{E}*P_{E}}{\Delta DUR_{Swap}}=\frac{(-2,25)*\$1.000.000}{(-4,5)}=\$500.000,00$

Wir sollten also in den Swap $\$500.000,00$ Investieren um unser
Interest Rate Risk zu minimieren.
\end{document}
