\author{Szymon Chlopek - XXXXXXXX \\ Christian Otto - XXXXXXXX}
\title{Derivate Case}
\date{\today}

\documentclass[12pt,a4paper,twoside]{article}
\usepackage{ngerman}
\usepackage [latin1]{inputenc}
\usepackage{graphicx}

\nonfrenchspacing

\begin{document}
\maketitle

\textbf{Aufgabenteil (i)}\\
\begin{tabular}{|l|r|r|r|r|r|}
  \hline
  \textbf{R�ckzahlungsdatum}&\textbf{1}&\textbf{2}&\textbf{3}&\textbf{4}&\textbf{5}\\
  \hline
  \textbf{12.08.2011}&0&0&0&107,5&0\\
  \hline
  \textbf{13.08.2012}&0&0&115&115&0\\
  \hline
  \textbf{12.08.2013}&0&122,5&122,5&122,5&0\\
  \hline
  \textbf{12.08.2014}&0&130&130&130&0\\
  \hline
  \textbf{12.08.2015}&0&137,5&137,5&137,5&137,5\\
  \hline
\end{tabular} \\
Die genauen Berechnungen sind dem Excel-Sheet zu entnehmen.\\ \\
\textbf{Aufgabenteil (ii)} \\ 
\begin{tabular}{|l|r|}
  \hline
  &\textbf{Wert}\\
  \hline
  \textbf{Risikoloser Zins} $r$&2,40 Prozent\\
  \hline
  \textbf{Volatilit�t}&32,54 Prozent\\
  \hline
\end{tabular} \\ \\
Hierbei wurden der Seite http://www.deutsche-finanzagentur.de am 19.01.2011 der Wert 2,40 Prozent entnommen. Dieser entspricht der Rendite einer kurzfristigen Bundesanleihe. F�r die Volatilit�t wurde die Funktion STABW benutzt, um den Sch�tzer der Standardabweichung zu finden. \\
Die genauen Berechnungen sind dem Excel-Sheet zu entnehmen.\\ \\
\textbf{Aufgabenteil (iii)} \\ 
\begin{tabular}{|l|r|r|}
  \hline
  $X_t$&\textbf{Kurs}&\textbf{R�ckzahlung}\\
  \hline
  \textbf{0,3}&2231,4219&0\\
  \hline
  \textbf{0,1}&1744,1533&0\\
  \hline
  \textbf{0,6}&2874,0171&122,5\\
  \hline
  \textbf{0,1}&1744,1533&130\\
  \hline
  \textbf{0,8}&3480,3386&137,5\\
  \hline
\end{tabular} \\
Die genauen Berechnungen sind dem Excel-Sheet zu entnehmen.\\ \\
\textbf{Aufgabenteil (iv)}\\
\textbf{Zertifkatswert}: 108,5736\\
Die genauen Berechnungen und Simulationen sind dem Excel-Sheet zu entnehmen.\\ \\
F�r die Monte-Carlo-Simulation sprechen einige Gr�nde.\\
So erm�glichen uns heute Computer durch Simulationen komplexe Sachverhalte auch in mehreren Szenarien schnell und kosteneffizient zu bestimmen. Zudem basieren die meisten Modelle auf mathematischen Formeln und Modellen, deren charakteristische Eigenschaften nur teilweise oder gar nicht in geschlossenen Formeln gefasst werden kann. Zudem f�hrt der Komplexit�tsgrad vieler Modelle dazu, dass diese nur noch schwer ohne Zuhilfenahme von Computersimulationen beschreibbar sind.\\
Im Gegensatz dazu, gibt es auch einige nicht von der Hand zu weisende Nachteile.\\
So f�hrt eine Verarbeitung durch einen Computer automatisch zu Ungenauigkeiten, die auf der beschr�nkten Darstellungskapazit�t der Datentypen beruhen und sich durch Weiterverwendung der Berechnungsergebnisse fortpflanzen und auch vergr��ern kann. Analog dazu f�hren auch Modellfehler/-ungenauigkeiten zu dem selben Problem, dass Abweichungen zur Realit�t entstehen, die sich fortpflanzen, je l�nger die Betrachtungsperiode w�hrt. Zudem spielen hier die Probleme herein, dass in der realen Welt nicht zwingend Bewertungen nur nach dem unterstellten System angestellt werden. Zudem sind die Ergebnisse von der Verteilung der Zufallszahlen abh�ngig. Jedoch sollte bei gen�gend gro�en Stichproben der statistische Fehler durch ungl�ckliche Ziehungen von Zufallszahlen gegen 0 laufen und sich stabilisieren. Zudem ist die Monte Carlo-Methode relativ aufw�ndig, auch wenn die Rechenzeit auf aktuellen Computern nur noch ein geringes Problem darstellt.\\ \\
Abschlie�end l�sst sich sagen, dass die Monte-Carlo-Simulation zwar unter einigen Nachteilen leidet, die die Prognosequalit�t schm�lern, jedoch die Vorteile �berwiegen, da einige der Modelle nicht oder nur schwer in geschlossene Formeln fassbar sind, und ansonsten eine Bewertung in diesen Modellen schwer fallen d�rfte.

\end{document}