\author{Szymon Chlopek - XXXXXXXX \\ Christian Otto - XXXXXXXX}
\title{Derivate Case}
\date{\today}

\documentclass[12pt,a4paper,twoside]{article}
\usepackage{ngerman}
\usepackage [latin1]{inputenc}
\usepackage{graphicx}
\nonfrenchspacing

\begin{document}
\maketitle

\textbf{Aufgabenteil (i)}
\begin{itemize}
	\item \textbf{Sch�tzung des Erwartungswertes sowie der Varianz} - siehe dazu die zugeh�rige Excel-Datei (derivatecase szymon chlopek.xlsx) bzw. die unten hinzugef�gte Tabelle aller errechneten Werte.
	\item \textbf{Bestimmung der Up und Down-Sp�rnge} unter der Annahme $u=1/d$ - hierf�r wendet man die Formel nach Cox/Rubinstein (Skript S.114), die da besagt: $u:= \exp(\sigma \sqrt(\frac{T}{n}))$, wobei $\sigma = \sqrt(Var)$ - die Standardabweichung, $T$ - die Dauer, $n$ - Anzahl der Spr�nge ist. Das zugrunde liegende Modell liefert: $\sigma = 0.1015, T=5, n=5.$ Setzt man dies in obige Gleichung, so erh�lt  man f�r $u=1.106779474$ sowie entsprechend $d=\frac{1}{u}=0.9035223579.$
\end{itemize}
\textbf{Aufgabenteil (ii)} \\ 
In diesem Teil rechnet man die Up-Wahrscheinlichkeiten unter dem physischen Ma� mit der Formel $p(n)=\frac{1}{2} + \frac{1}{2}\frac{\mu}{\sigma}\sqrt{\frac{T}{n}}$. Weiter setzt man die oben gennate Werte sowie $\mu= 0.0019$ ein und bekommt $p=0.5095$. Weiter wird die Formel $q=\frac{\exp(iT)-d}{u-d}$ beznutzt um die Up-Wahrscheinlichkeiten unter dem risikoneutralen Ma� auszurechnen. Hierbei ergibt sich $q=0.7269.$\\
\\
\textbf{Aufgabenteil (iii) sowie (iv)} \\
Programmierung eines volst�ndigen Baumes �ber 5 Peridoden f�r das risikoneutrale WMa� und Strike in H�he von $30,35,40$ - siehe hierf�r die dazugeh�rige Excel-Datei. \\
\\
\textbf{Aufgabenteil (v)}
\begin{itemize}
	\item Bewertung eines amerikanischen Call (long) - Strike $K = 35$, Laufzeit 5 Monate in den Knoten \textsl{d-u-u (in t=3), u-u-d-d (in t=4), u-u-u-u (in t=4)} - siehe Excel
	\item Wie der Excel-Datei zu entnehmen ist, ist es in keinem der angegebenen Knoten sinnvoll das Derivat auszu�ben. Immer ist es sinnvoll bis zu F�lligkeitsdatum zu warten.  \\
	\item \textbf{Aus�bung sobald der Payoff positiv - Bewertung}: Bei dieser Strategie �bt man sofort aus, sobald ein positiver Payoff erreicht wird. Da die Vorteilhaftigkeit nicht betrachtet wird, sinkt das erzielbare Ergebnis im Vergleich zu amerikanischem Call. Nach M�glichkeit wird bereits in $t=0$ ausge�bt.
\end{itemize}
\textbf{Aufgabenteil vi} - \textbf{Wert eines amer. Calls bei einer Dividende}
\begin{itemize}
	\item Dividende nach der 4 Periode in H�he von 5 Euro - Der Wert des amerikanischen Calls betr�gt 7.9891 Euro (Rechnung ist der Excel-Datei zu entnehmen.
	\item Dividende nach der 4 Periode in H�he von $0.5\%$ - Der Wert des amerikanischen Calls betr�gt 11.5111 Euro (Rechnung ist der Excel-Datei zu entnehmen.
\end{itemize}


\par\medskip
\begin{center}

\begin{tabular}{|l|l|l|l|l|l|} 

\hline
\textbf{EW} & \textbf{Var} &  \textbf{Up-Wkeit} & \textbf{Down-Wkeit} & \textbf{Phy. Ma� p} &\textbf{Ris.Neu. Ma� q} \\
\hline
\hline
0,0019 & 0.0103 & 1.1068 & 0.9035 & 0.5095 & 0.7269 \\
\hline
\end{tabular}
\end{center}


\par\medskip
\begin{center}

\begin{tabular}{|l|l|l|l|l|l|l|} 

\hline
\textbf{$C^{e}$ Strike 30} & \textbf{Strike 35} &  \textbf{Strike 40} & \textbf{$C^{a}$} & Payoff & \textbf{Div = 5} &\textbf{Div = 0.05} \\
\hline
\hline
15,3174 & \textbf{11,5147} & 8,1244 & \textbf{11,5147} & 3,6500 & 10,5381 & 11,3384 \\
\hline
\end{tabular}
\end{center}

\end{document}