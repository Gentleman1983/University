\documentclass[a4paper,10pt]{article}

\usepackage[latin1]{inputenc}
\usepackage[ngerman]{babel}
\usepackage{amsthm}
\usepackage{amsfonts}
\usepackage{enumitem}

\theoremstyle{definition}
    \newtheorem{defi}[equation]{Definition}
\theoremstyle{remark}
    \newtheorem{bsp}[equation]{Beispiel}
    \newtheorem*{bew}{Beweis}
\theoremstyle{plain}
    \newtheorem{hpts}[equation]{Hauptsatz}
    \newtheorem*{sbw}{Satz von Bolzano-Weierstra\ss}

\setlist{noitemsep, label=(\arabic*)}

\begin{document}

    \begin{defi}
        Ein geod\"atischer metrischer Raum hei{\ss}t \textit{CAT(0)-Raum}, wenn jedes Dreieck
        in ihm d\"unner ist als ein Dreieck mit denselben Seitenl\"angen im Euklidischen Raum.
    \end{defi}

    \begin{bsp}
        Die folgenden R\"aume sind
        CAT(0)-R\"aume:\begin{enumerate}
                \item B\"aume, versehen mit der nat\"urlichen
                Wegmetrik;
                \item einfach zusammenh\"angende, vollst\"andige Riemannsche Mannigfaltigkeiten
                nicht-positiver Kr\"ummung;
                \item affine Geb\"aude.

        \end{enumerate}
        Insbesondere ist der gew\"ohnliche Euklidische Raum CAT(0).
    \end{bsp}

    \begin{hpts}
        Eine kompakte Gruppe wirke stetig und isometrisch auf einem lokal kompakten CAT(0)-Raum.
        Dann gibt es einen gemeinsamen Fixpunkt.
    \end{hpts}

    \begin{proof}
        Der Beweis zerf\"allt in mehrere Schritte.
        \begin{enumerate}
            \item W\"ahle einen beliebigen Punkt im CAT(0)-Raum und betrachte seinen Orbit. Dies
            ist eine kompakte Teilmenge. Betrachte die Menge aller Kugeln, die diesen Orbit
            enthalten.
            \item Wir behaupten, dass es \textit{genau eine} solche Kugel gibt, deren Radius minimal
            ist.
            \item Um \textit{Existenz} von Kugeln mit minimalem Radius zu beweisen, w\"ahle eine Folge
            von Kugeln, deren Radien gegen die untere Schranke f\"ur die Radien konvergieren. Ihre
            Mittelpunktenliegen alle in einer kompakten Teilmenge, weil unser Raum lokal kompakt und
            vollst\"andig ist. Nach Auswahl einer Teilfolge konvergieren die Mittelpunkte gegen einen
            Punkt in unserem Raum. Er ist der Mittelpunkt eines Kreises von minimalem Radius.
            \item Um \textit{Eindeutigkeit} von Kugeln mit minimalem Radius zu beweisen, nehme an, es
            g\"abe mehr als eine solche Kugel. Dann ist unsere kompakte Teilmenge im Durchschnitt zweier
            Kugeln von gleichem Radius enthalten. Ein geometrisches Argument zeigt, dass dieser
            Durchschnitt schon in einer kleineren Kugel enthalten ist, was Minimalit\"at widerspricht.
            \item Der Mittelpunkt der Kugel mit kleinstem Radius, die den Orbit enth\"alt, ist unter der
            Gruppenwirkung invariant, weil sie ihn wieder auf Mittelpunkte von minimalen Kugeln abbildet.
            \qedhere
        \end{enumerate}
    \end{proof}

    \begin{sbw}
        Jede beschr\"ankte unendliche Menge reeller Zahlen besitzt einen H\"aufungspunkt.
    \end{sbw}

\end{document}
