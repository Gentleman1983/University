\documentclass{article}
\usepackage[utf8]{inputenc}
\usepackage[ngerman]{babel}
\usepackage{amsthm}
\theoremstyle{plain}
\newtheorem{theorem}[equation]{Hauptsatz}
\newtheorem*{theo}{Satz}
\theoremstyle{definition}
\newtheorem{definition}[equation]{Definition}
\theoremstyle{remark}
\newtheorem{example}[equation]{Beispiel}
\begin{document}
\begin{definition}
Ein geod"atischer metrischer Raum hei"st ein \emph{CAT(0)-Raum}, wenn
jedes Dreieck in ihm d"unner ist als ein Dreieck mit denselben Seitenl"angen im
Euklidischen Raum.
\end{definition}

\begin{example}
Die folgenden R"aume sind CAT(0)-R"aume:\\
(1) B"aume, versehen mit der nat"urlichen Wegmetrik;\\
(2) einfach zusammenh"angende, vollst"andige Riemannsche Mannigfaltigkeiten
nicht-positiver Kr"ummung;\\
(3) affine Geb"aude.\\
Insbesondere ist der gew"ohnliche Euklidische Raum CAT(0).
\end{example}

\begin{theorem}
Eine kompakte Gruppe wirke stetig und isometrisch auf einem lokal
kompakten CAT(0)-Raum. Dann gibt es einen gemeinsamen Fixpunkt.
\end{theorem}

\begin{proof}
Der Beweis zerf"allt in mehrere Schritte.\\
(1) W"ahle einen beliebigen Punkt im CAT(0)-Raum und betrachte seinen Orbit.
Dies ist eine kompakte Teilmenge. Betrachte die Menge aller Kugeln,
die diesen Orbit enthalten.\\
(2) Wir behaupten, dass es \emph{genau eine} solche Kugel gibt, deren Radius minimal
ist.\\
(3) Um \emph{Existenz} von Kugeln mit minimalem Radius zu beweisen, w"ahle eine
Folge von Kugeln, deren Radien gegen die untere Schranke f"ur die Radien
konvergieren. Ihre Mittelpunkte liegen alle in einer kompakten Teilmenge,
weil unser Raum lokal kompakt und vollst"andig ist. Nach Auswahl einer
Teilfolge konvergieren die Mittelpunkte gegen einen Punkt in unserem
Raum. Er ist der Mittelpunkt eines Kreises von minimalem Radius.\\
(4) Um \emph{Eindeutigkeit} von Kugeln mit minimalem Radius zu beweisen, nehme
an, es g"abe mehr als eine solche Kugel. Dann ist unsere kompakte Teilmenge
im Durchschnitt zweier Kugeln von gleichem Radius enthalten. Ein geometrisches
Argument zeigt, dass dieser Durchschnitt schon in einer kleineren
Kugel enthalten ist, was der Minimalit"at widerspricht.\\
(5) Der Mittelpunkt der Kugel mit kleinstem Radius, die den Orbit enth"alt, ist
unter der Gruppenwirkung invariant, weil sie ihn wieder auf Mittelpunkte
von minimalen Kugeln abbildet. 
\end{proof}

\begin{theo}[Bolzano-Weierstra"s]
Jede beschr"ankte unendliche Menge reeller Zahlen besitzt einen H"aufungspunkt.
\end{theo}
\end{document}


