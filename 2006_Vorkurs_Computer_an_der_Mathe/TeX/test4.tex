\documentclass[a4paper,12pt]{article}

\usepackage[latin1]{inputenc}
\usepackage{amsfonts}
\usepackage{amsmath}
\usepackage{amssymb}
\usepackage{amsthm}
\usepackage[ngerman]{babel}
\usepackage[T1]{fontenc}
\usepackage{graphicx}


\theoremstyle{plain}
\newtheorem{thm}{Theorem}[section]

\begin{document}
\tableofcontents
\section{Der Satz vom kleinen Moritz}
\begin{thm}
Sei $K$ ein kommutativer K�rper mit $1+1 = 0$. Dann gilt 
\begin{displaymath}
(x + y)^2 = x^2 + y^2
\end{displaymath}
f�r alle $x,y \in K$.
\end{thm}
\begin{proof}
Es gilt: 
\begin{align}
(x+y)^2 &= (x+y)(x+y)
&= x^2 + xy + xy + y^2\\
&= x^2 + (1+1)xy + y^2\\
&= x^2 + y^2
\end{align}
\end{proof}

\section{Ein kleiner Beweis der Riemann-Vermutung}
 
In Vorbereitung \ldots

\end{document}